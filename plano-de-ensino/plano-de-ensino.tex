\documentclass{UnBExam}%
\usepackage{url}
\usepackage[brazilian]{babel}
\usepackage{amsmath}
\disciplina{Treinamento para Competições de Programação}%
\turma{}%
\documento{}%
\subdoc{Plano de ensino}%
\periodo{}%
\professora{}
\newcommand{\numerocontests}{3}
\newcommand{\fatorescala}{150}

\setlength\parindent{0pt}% 	remover tabulação

% Mudar estilo da seção
\makeatletter%
\renewcommand\section{\@startsection{section}{1}{\z@}%
	{-3.5ex \@plus -1ex \@minus -.2ex}%
	{2.3ex \@plus.2ex}%
	{\normalfont\normalsize\bfseries}}%
\makeatother%

\printanswers

\renewcommand{\indent}{\hspace{.5cm}}

\begin{document}

\section{Objetivo da disciplina}
\indent Esta disciplina recursável tem como objetivo incentivar os alunos da UnB a participarem de competições de programação.

\vspace{-.1cm}
\section{Procedimento de ensino}
\indent Não haverão aulas ou avaliações presenciais. O aluno deve utilizar recursos de aprendizado e treinamento autodidatas, como livros \cite{halim2013competitive,cormen2009introduction} e a Internet.

\vspace{-.1cm}
\section{Avaliação do aluno}
\indent Até o fim do \textbf{período de aulas} do \textbf{calendário universitário de graduação} (alterações feitas no calendário durante o período não afetam esta disciplina e, portanto, não serão consideradas), o aluno deve realizar as seguintes tarefas (caso contrário receberá menção final SR):
\begin{enumerate}
	\item Registrar-se na plataforma Codeforces (\url{http://codeforces.com/register}).
	\item Preencher corretamente as informações sociais e atribuir \texttt{University of Brasilia} ao campo \texttt{Organization} (\url{http://codeforces.com/settings/social}).
	\item Enviar seu nome completo, matrícula e nome de usuário no Codeforces (campo \texttt{handle}) para \url{treinopcunb@gmail.com}.
	\item Ler as regras das competições do Codeforces (\url{http://codeforces.com/blog/entry/4088}).
	\item Participar de \textbf{no mínimo \numerocontests} competições com \textit{rating}, ou seja, daquelas que ficam registradas no gráfico de desempenho do usuário (exemplo: \url{http://codeforces.com/profile/tourist}).
\end{enumerate}
\indent Sejam $S_1 \geq S_2 \geq \cdots \geq S_n$ os \textit{scores} obtidos nas $n$ competições registradas no gráfico do aluno durante o período de aulas, em ordem não-crescente (apenas \textit{score} obtido por solucionar problemas é considerado!). Então

\vspace{-.8cm}
\[
\begin{array}{ccc}
	N = \dfrac{1}{\fatorescala} \dfrac{\sum_{i=1}^{\numerocontests} S_i}{\numerocontests} & \;\;\;\;\; \text{e} \;\;\;\;\; &
	\text{Menção final} =
	\begin{cases}
	\hfill \text{SR} \hfill & \text{se } N \in [0,0.1)   \\
	\hfill \text{II} \hfill & \text{se } N \in [0.1,3)   \\
	\hfill \text{MI} \hfill & \text{se } N \in [3,5)     \\
	\hfill \text{MM} \hfill & \text{se } N \in [5,7)     \\
	\hfill \text{MS} \hfill & \text{se } N \in [7,9)     \\
	\hfill \text{SS} \hfill & \text{se } N \geq 9        \\
	\end{cases}
\end{array}
\]

\vspace{-1.8cm}
\section{Informações úteis}
\vspace{-.2cm}
\indent -- Listas de exercícios: \url{https://a2oj.com/ladders}

\indent -- Habilite a opção \texttt{Contest email notification} para receber por e-mail notícias sobre novas competições (\url{http://codeforces.com/settings/general}).

\indent -- Geralmente, após o término da competição, os autores postam editoriais com as soluções dos problemas. Fique de olho nas postagens recentes!

\indent -- As menções finais serão calculadas com a seguinte calculadora: \url{http://goo.gl/7n4N7m}

\renewcommand\refname{\vskip -1cm}

\vspace{-.1cm}
\section{Bibliografia recomendada}
\bibliographystyle{unsrt}
\bibliography{plano-de-ensino}
\end{document}