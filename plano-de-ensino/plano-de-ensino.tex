\documentclass{UnBExam}%
\usepackage{url}
\usepackage[brazilian]{babel}
\usepackage{amsmath}
\disciplina{Treinamento para Competições de Programação}%
\turma{}%
\documento{}%
\subdoc{Plano de ensino}%
\periodo{}%
\professora{Carla Denise Castanho}
\newcommand{\numerocontests}{3}
\newcommand{\fatorescala}{100}

\setlength\parindent{0pt}% 	remover tabulação

% Mudar estilo da seção
\makeatletter%
\renewcommand\section{\@startsection{section}{1}{\z@}%
	{-3.5ex \@plus -1ex \@minus -.2ex}%
	{2.3ex \@plus.2ex}%
	{\normalfont\normalsize\bfseries}}%
\makeatother%

\printanswers

\renewcommand{\indent}{\hspace{.5cm}}

\begin{document}

\section{Objetivo da disciplina}
\vspace{-.25cm}
\indent Incentivar alunos da UnB a participarem de competições de programação.

\vspace{-.25cm}
\section{Procedimento de ensino}
\vspace{-.25cm}
\indent Não haverão aulas ou avaliações presenciais. O aluno deve utilizar recursos de aprendizado e treinamento autodidatas, como bibliotecas públicas e a Internet \cite{halim2013competitive,cormen2009introduction,pimentaarticle}.

\vspace{-.25cm}
\section{Avaliação do aluno}
\vspace{-.25cm}
\indent Até o fim do \textbf{período de aulas} do calendário universitário \textbf{original} (alterações no calendário feitas no decorrer do período não afetam este curso!), o aluno deve realizar as seguintes tarefas (caso contrário receberá menção SR):
\begin{enumerate}
	\item Registrar-se na plataforma Codeforces (\url{http://codeforces.com/register}).
	\item Preencher corretamente todas as informações sociais e atribuir \texttt{University of Brasilia} ao campo \texttt{Organization} (\url{http://codeforces.com/settings/social}).
	\item Enviar para \url{carladenisecastanho@gmail.com} seu nome de usuário no Codeforces (o campo \texttt{handle}).
	\item Ler as instruções e regras das competições (\url{http://codeforces.com/blog/entry/4088}).
	\item Participar de \textbf{no mínimo \numerocontests} competições com \textit{rating} (competições que são registradas em seu gráfico de desempenho. Exemplo: \url{http://codeforces.com/profile/tourist}).
\end{enumerate}
\indent Seja $M$ a média aritmética dos \numerocontests\ maiores \textit{scores} (apenas \textit{score} obtido por solucionar problemas é considerado!) registrados no gráfico do aluno durante o período de aulas e seja $N = M/\fatorescala$. Então

\vspace{-.2cm}
\[
\begin{array}{ccc}
	\text{Menção final} =
	\begin{cases}
	\hfill \text{SR} \hfill & \text{se } N \in [0,0.1)   \\
	\hfill \text{II} \hfill & \text{se } N \in [0.1,3)   \\
	\hfill \text{MI} \hfill & \text{se } N \in [3,5)     \\
	\hfill \text{MM} \hfill & \text{se } N \in [5,7)     \\
	\hfill \text{MS} \hfill & \text{se } N \in [7,9)     \\
	\hfill \text{SS} \hfill & \text{se } N \in [9,\infty)\\
	\end{cases}
\end{array}
\]

\vspace{-1cm}
\section{Informações úteis}
\vspace{-.25cm}
\indent -- A opção ``Contest email notification'' (\url{http://codeforces.com/settings/general}) faz com que a plataforma avise por e-mail sobre competições que estão por vir.

\indent -- Geralmente, os autores de uma competição postam um editorial com as soluções dos problemas após o término da competição. Fique atento às postagens mais recentes!

\indent -- As menções finais serão calculadas com a seguinte calculadora: \url{http://goo.gl/7n4N7m}

\renewcommand\refname{\vskip -1cm}

\vspace{-.25cm}
\section{Bibliografia recomendada}
\bibliographystyle{unsrt}
\bibliography{plano-de-ensino}
\end{document}