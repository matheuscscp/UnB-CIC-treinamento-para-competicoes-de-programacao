\documentclass{UnBExam}%
\usepackage{url}
\usepackage[brazilian]{babel}
\usepackage{amsmath}
\disciplina{Treinamento para Competições de Programação}%
\turma{}%
\documento{}%
\subdoc{Plano de ensino}%
\periodo{}%
\professora{Célia Ghedini Ralha}
\newcommand{\dataprimeirodia}{11 de agosto de 2015}

\setlength\parindent{0pt}% 	remover tabulação

% Mudar estilo da seção
\makeatletter%
\renewcommand\section{\@startsection{section}{1}{\z@}%
	{-3.5ex \@plus -1ex \@minus -.2ex}%
	{2.3ex \@plus.2ex}%
	{\normalfont\normalsize\bfseries}}%
\makeatother%

\printanswers

\renewcommand{\indent}{\hspace{.5cm}}

\begin{document}
\section{Objetivo da disciplina}%
\vspace{-.25cm}
\indent Incentivar alunos da UnB a participarem de competições de programação.
\vspace{-.3cm}
\section{Procedimento de ensino}%
\vspace{-.2cm}
\indent O aluno deve utilizar recursos de aprendizado e treinamento autodidatas, como, por exemplo, bibliotecas públicas \cite{halim2013competitive,Skiena03,cormen2009introduction,Skiena08,Dasgupta06,pimentabook,pimentaarticle} e a Internet.
\vspace{-.3cm}
\section{Avaliação do aluno}%
\vspace{-.25cm}
\indent O aluno deve:
\begin{enumerate}
	\item se registrar na plataforma Codeforces pelo endereço \url{http://codeforces.com/register};
	\item enviar um e-mail contendo seu nome completo, matrícula e nome de usuário na plataforma Codeforces (\emph{handle}) para o endereço \url{ghedini@unb.br}, com o assunto ``Handle no Codeforces'';
	\item participar, durante o período letivo vigente da UnB, de no mínimo três competições com \textit{rating}, ou seja, competições que apareçam no gráfico de desempenho do usuário.\\
	Exemplo de gráfico: \url{http://codeforces.com/profile/tourist}
\end{enumerate}
\indent O aluno que não realizar as etapas acima receberá menção final SR.

\indent Sejam $P_1,P_2,...,P_n$ as pontuações de um aluno nas $n$ competições com \textit{rating} das quais ele participou no período vigente. Então

\vspace{-.5cm}
\begin{minipage}{8cm}
	$$N = \frac{\sum_{i=1}^{n} P_i}{100n}$$
\end{minipage}
\begin{minipage}{8cm}
	\[
	\text{Menção final} =
	\begin{cases}
	\hfill \text{SR} \hfill & \text{se } 0 \leq N < 0.1 \\
	\hfill \text{II} \hfill & \text{se } 0.1 \leq N < 3 \\
	\hfill \text{MI} \hfill & \text{se } 3 \leq N < 5 \\
	\hfill \text{MM} \hfill & \text{se } 5 \leq N < 7 \\
	\hfill \text{MS} \hfill & \text{se } 7 \leq N < 9 \\
	\hfill \text{SS} \hfill & \text{se } 9 \leq N \\
	\end{cases}
	\]
\end{minipage}
\vspace{-.75cm}
\section{Informações úteis}
\vspace{-.15cm}
\indent -- As regras das competições estão no endereço \url{http://codeforces.com/blog/entry/4088}.

\indent -- A plataforma Codeforces avisa por e-mail sobre competições que estão por vir.

\indent -- Editoriais com as soluções dos problemas são publicados na plataforma após cada competição.
\renewcommand\refname{\vskip -1cm}
\vspace{-.75cm}
\section{Bibliografia recomendada}%
\bibliographystyle{unsrt}
\bibliography{plano-de-ensino}
\end{document}