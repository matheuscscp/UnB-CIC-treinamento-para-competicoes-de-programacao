\documentclass{UnBExam}%
\usepackage{url}
\usepackage[brazilian]{babel}
\usepackage{amsmath}
\disciplina{Treinamento para Competições de Programação}%
\turma{}%
\documento{}%
\subdoc{Plano de ensino}%
\periodo{}%
\professora{Célia Ghedini Ralha}
\newcommand{\dataprimeirodia}{11 de agosto de 2015}
\newcommand{\numerocontests}{3}
\newcommand{\fatorescala}{100}

\setlength\parindent{0pt}% 	remover tabulação

% Mudar estilo da seção
\makeatletter%
\renewcommand\section{\@startsection{section}{1}{\z@}%
	{-3.5ex \@plus -1ex \@minus -.2ex}%
	{2.3ex \@plus.2ex}%
	{\normalfont\normalsize\bfseries}}%
\makeatother%

\printanswers

\renewcommand{\indent}{\hspace{.5cm}}

\begin{document}
	
\section{Objetivo da disciplina}
\indent Incentivar alunos da UnB a participarem de competições de programação.

\section{Procedimento de ensino}
\indent Não haverão aulas ou avaliações presenciais. O aluno deve utilizar recursos de aprendizado e treinamento autodidatas, como bibliotecas públicas e a Internet \cite{halim2013competitive,cormen2009introduction,pimentaarticle}.

\section{Avaliação do aluno}
\indent Até o fim do período de aulas do calendário universitário, o aluno deve:
\begin{enumerate}
	\item Registrar-se na plataforma Codeforces (\url{http://codeforces.com/register}).
	\item Preencher corretamente todas as informações sociais e atribuir ``University of Brasilia'' ao campo ``Organization'' (\url{http://codeforces.com/settings/social}).
	\item Enviar para \url{ghedini@unb.br} seu nome de usuário no Codeforces (o campo \textit{handle}).
	\item Participar de no mínimo três competições com \textit{rating}, ou seja, competições que são registradas em seu gráfico de desempenho (exemplo: \url{http://codeforces.com/profile/tourist}).
\end{enumerate}
\indent Seja $N$ a média aritmética dos \numerocontests\ maiores \textit{scores} (apenas \textit{score} obtido por solucionar problemas é considerado) registrados no gráfico do aluno durante o período de aulas (se há menos do que \numerocontests\ competições registradas a menção final é SR) dividida por \fatorescala. Então

\vspace{-.3cm}
\[
\begin{array}{ccc}
	\text{Menção final} =
	\begin{cases}
	\hfill \text{SR} \hfill & \text{se } 0 \leq N < 0.1 \\
	\hfill \text{II} \hfill & \text{se } 0.1 \leq N < 3 \\
	\hfill \text{MI} \hfill & \text{se } 3 \leq N < 5 \\
	\hfill \text{MM} \hfill & \text{se } 5 \leq N < 7 \\
	\hfill \text{MS} \hfill & \text{se } 7 \leq N < 9 \\
	\hfill \text{SS} \hfill & \text{se } 9 \leq N \\
	\end{cases}
\end{array}
\]

\vspace{-1.6cm}
\section{Informações úteis}
\indent -- Há uma postagem com as regras das competições (\url{http://codeforces.com/blog/entry/4088}).

\indent -- A opção ``Contest email notification'' (\url{http://codeforces.com/settings/general}) faz com que a plataforma avise por e-mail sobre competições que estão por vir.

\indent -- Geralmente, os autores de uma competição postam um editorial com as soluções dos problemas após o término da competição. Fique atento às postagens mais recentes!

\indent -- As menções finais serão calculadas com a seguinte calculadora: \url{http://goo.gl/7n4N7m}

\renewcommand\refname{\vskip -1cm}
\section{Bibliografia recomendada}
\bibliographystyle{unsrt}
\bibliography{plano-de-ensino}
\end{document}